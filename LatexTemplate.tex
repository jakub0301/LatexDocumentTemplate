\documentclass[12pt,a4paper]{article}
\usepackage[a4paper, left=2cm, right=2cm, top=2cm, bottom=2cm]{geometry} 
\usepackage[final]{graphicx}
\usepackage{subcaption}
\usepackage[export]{adjustbox}
\usepackage{amsmath,float,bm}
\usepackage[affil-it]{authblk}
\usepackage{titlesec,wrapfig,amssymb}
\usepackage{empheq}
\usepackage{amsthm}
\usepackage[T1]{fontenc}
\usepackage[polish]{babel}
\usepackage[utf8]{inputenc}
\usepackage{graphicx}
\usepackage{hyperref}
\graphicspath{ {/}}


\begin{document}
\newtheorem{fakt}{Fakt}
\begin{flushleft}
Name\\
Adress\\
Mail\\
EXAMPLE\\

\end{flushleft}\


\section*{Zadanie 7}



\begin{proof}[Rozwiązanie zadania 7]
\includegraphics[width=0.7\textwidth]{zad7}

Niech punkty $A=(0,0)$, $B=(1,0)$, $C=(c,h)$ oraz $D=(d,h)$
Niech punkt $E=AC \cap BD$
Zatem jest on przecięciem funkcji liniowych danymi wzorami $y=\frac{h}{c}x$ oraz $y=\frac{h}{d-1}(x-1)$
\\ $\frac{h}{c}x=\frac{h}{d-1}-\frac{h}{d-1}$
\\$x(\frac{h}{c}-\frac{h}{d-1}=-	\frac{h}{d-1})$
\\$x(d-1-c)=-c$
\\Zatem $x_{E}=\frac{c}{1-d+c} oraz \ y_{E}=\frac{h}{1-d+c}$
\\ Niech $C_{0}$ będzie rzutem punktu C na prostą $AD$ 
$$\left\{\begin{array}{rcl}

y=&\frac{h}{d}x \\
y-h=& -\frac{d}{h}(x-c) \\
\end{array} \right.$$

\end{proof}

\end{document}

